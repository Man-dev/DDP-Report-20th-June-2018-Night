\chapter{Conclusion and Future Work}
The objective of this study is to include ray-optics in the heat loss analysis of a trapezoidal cavity receiver. And eventually couple ray-optics with heat transfer model using a single software COMSOL Multiphysics 5.3. Specifically, to develop a coupled model of ray optics with heat transfer for LFR-TCR system in which the software calculates heat received by all the surfaces inside the trapezoidal cavity receiver after performing ray tracing simulation on combined system of linear Fresnel reflector and trapezoidal cavity receiver and then automatically starts performing heat loss analysis in the cavity. Although automatic coupling of both the models has not been achieved but coupling has been done in COMSOL Multiphysics 5.3 which requires some manual effort from the user so that the output of the ray optics simulation is passed into the heat transfer model. Fully automated coupling can be the next attempt as continuation of the current study.

A lot of previous research on the heat loss study from trapezoidal cavity receiver has been done under the assumption that the pipes carrying the working fluid are at a uniform temperature. Current study does not begin with the same assumption and attempts to examine the validity of the assumption. It has been observed that as the mass flow rate of the working fluid flowing inside the pipes increases, the temperature of pipes begin to converse to a uniform temperature. Although point conversion of pipe temperatures may not be achieved but temperature difference among pipes becomes very low. Thus, this study concludes that the assumption of uniform temperature of pipes is valid to a very good degree of accuracy at higher flow rates.

In the present study, heat absorbed by the pipes carrying the working fluid and cavity walls inside trapezoidal cavity receiver is calculated by ray optics simulation. In this simulation, deterministic approach is taken over Monte Carlo ray tracing as the geometry is quite simple and it is a 2-D analysis. So, any significant advantage of Monte Carlo ray tracing was not felt. Although, the comparison of the heat falling on the pipes inside the trapezoidal cavity receiver as calculated by deterministic approach against Monte Carlo ray tracing approach has not been done in the current study. In the future, by continuing the current study this comparison between the two approaches would be a good thing to do.

Present study models heat transfer via convection by conduction only and advection term is neglected. Results obtained in the present study show that this assumption is quite valid to a high degree of accuracy. However, the geometry used in the present study is hampering formation of convective currents. There are 8 pipes used and these are spread along entire with of the cavity. There is no space available to the left or to the right side of the cavity for convective currents to form. If same trapezoidal cavity receiver is used with just 4 pipes that will leave enough space for the formation of convective currents and then maybe neglecting advection term might not be a very good assumption. That is something which should be examined in the future.