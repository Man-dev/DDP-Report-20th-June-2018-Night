\chapter{Literature Survey} \label{ch2}

%Reconfiguration problem dealt with in distribution systems
%MG evolved from distribution systems
Microgrid reconfiguration has evolved from distribution system reconfiguration, or distribution reconfiguration. Now distribution reconfiguration has been around for quite a while, and we can find mention of distribution system reconfiguration as back as 1975\citep{Merl1975}.
%[    A. Merlin, H. BackSearch for a minimal-loss operating spanning tree configuration in an urban power distribution systemProc. 5th Power System Computation Conf, Cambridge (U.K.) (1975), pp. 1-18].
Since then, there has been quite extensive research on reconfiguration of distribution systems (\citep{mgrrev01},\citep{mgrrev02},\citep{mgrrev03}). There are certainly a lot of similar elements which microgrid reconfiguration draws from distribution system reconfiguration. Distribution system optimization is also cast as an optimization problem. The objectives used are quite similar, there are quite a lot of papers on power loss minimization (see \citep{mgrrev01}), and operating cost minimization (like \citep{mgrj20}) etc. but there are a few objectives not very prominently seen in microgrid reconfiguration, like energy not supplied (ENS)(\citep{mgrj25}), and indices like loadability index (\citep{mgrj27}) and SAIFI(\citep{mgrj25}), which talk about reliability of distribution systems. Also, most of the recent papers consider the distribution system optimization problem as a multiobjective problem, and either implement it as a weighted sum (\citep{mgrj25}) or use a multiobjective solvers such as NSGA 2(\citep{mgrj27},\citep{mgrj61}), and then rank the solutions obtained. The other papers considering single objective functions use methods similar to microgrid reconfiguration. The constraints consider are also very similar in all the cases. In addition, the recent researches consider distribution system reconfiguration along with some other decision variables, like dynamic reconfiguration in which the periods when reconfiguration happens are not fixed (\citep{mgrj20}), or demand response (\citep{mgrj28}), or siting and sizing of DERs (\citep{mgrj43}) etc. The load and generation profile is generally assumed to be static (taking a standard profile and ignoring variations) or forecasted profile is used (\citep{mgrj20}). So, overall there is a lot of similarity between the microgrid reconfiguration and distribution reconfiguration.\\
%[mgrrev01,mgrrev02,mgrrev03 - check if all three are required]
However, microgrid reconfiguration is a relatively new topic, and is in some aspects different from distribution system reconfiguration. One of the differences is the higher amount of DERs in microgrid, especially those having an uncertain nature (like solar or wind). In the earlier researches on reconfiguration in distribution system, these concepts are completely absent, and while recent researches include such DERs, usually the penetration of DERs in microgrids is quite high and hence the uncertainty which comes with it must be considered. But the biggest difference is the ability to island and operate independently from the main grid. This is really not present in the distribution system, so even if there are DERs, they are not necessarily supposed to drive the entire system, there's always main grid (or transmission system) to take care of the deficit in power balance. Also, once the system is islanded, the power quality and reliability is much more vital, since a large part of generation in most microgrids is DERs based off renewables, which may be uncertain.
%What generally has been reconf done for (objectives) general methods used
%practical applications if any
%What is different in MGs
%usually high ren - uncertainty
%local supply - ability to island - power balance and reliability is much more important
%mathematically how reliability and islanding yield differently

%Overview/methodology followed while searching
This review mainly focuses on the journal papers published related to microgrid reconfiguration. Most of the papers were presented treating reconfiguration as a decision variable/tool used for optimization. So while reviewing the papers, the stress is given to the specific objective function used and what it is optimizing, the constraints used in the problem, and the method used to achieve the optimal value. Table ~\ref{tab:litsurv} presents this analysis.\\
\begin{landscape}


\begin{table}
\centering
\tiny
\caption{Literature Survey of the Papers Related to Microgrid}
\label{tab:litsurv}
\begin{tabular}{|l|p{0.755cm}|p{0.755cm}|p{0.755cm}|p{0.755cm}|p{0.755cm}|p{0.755cm}|p{0.755cm}|p{0.755cm}|p{0.755cm}|p{0.755cm}|p{0.755cm}|p{0.755cm}|p{0.755cm}|p{0.755cm}|p{0.755cm}|p{0.755cm}|p{0.755cm}|p{0.755cm}|p{0.755cm}|p{0.755cm}|}
\hline
Number & Total Energy Losses in a period & Vulnerability index & Supplying Maximum \& High Priority Load & Operating Cost & Loadability or VSI & Additional Considerations & Radial Network & Power Flow Equation & Thermal Current Limits of the branches & Node Voltage Limits & Generation Limits & Load Generation Balance & Limit on Number of switching & Frequency Limits & Other Constraints                     & NSGA 2 Modified & GA & PSO & Others     & Uncertainty Model \\ \hline
\citep{mgrj07}      &                                 &                     &                                         & \checkmark              &                    &                           &                &                     & \checkmark                                      & \checkmark                   & \checkmark                 & \checkmark                       &                              &                  &                                       &                 & \checkmark  & \checkmark   & AIS, V-AIS & Fixed             \\ \hline
\citep{mgrj57}     &                                 & \checkmark                   &                                         &                &                    &                           & \checkmark              & \checkmark                   & \checkmark                                      & \checkmark                   & \checkmark                 &                         &                              &                  &                                       &                 & \checkmark  & \checkmark   & ICBA       & Fixed             \\ \hline
\citep{mgrj03}      &                                 &                     & \checkmark                                       &                &                    & Load Shedding             &                &                     &                                        &                     & \checkmark                 & \checkmark                       &                              &                  &                                       &                 & \checkmark  &     &            & Fixed             \\ \hline
\citep{mgrj02}      & \checkmark                               &                     &                                         & \checkmark              &                    & Storage                   &                & \checkmark                   & \checkmark                                      &                     &                   &                         &                              &                  & Probability of feeding all loads high &                 &    &     & LP         & Scenarios         \\ \hline
\citep{mgrj16}     &                                 & \checkmark                   &                                         &                &                    & Storage                   & \checkmark              & \checkmark                   & \checkmark                                      & \checkmark                   &                   &                         & \checkmark                            & \checkmark                &                                       &                 & \checkmark  & \checkmark   & APO,SV-APO & Fixed             \\ \hline
\citep{mgrj52}     &     \checkmark                            &                     &                                         & \checkmark              &                    &                           &                &                     &                                        & \checkmark                   &                   &                         & \checkmark                            &                  &                                       &                 &    &     & Rule Based & Fixed             \\ \hline
\citep{mgrj51}     & \checkmark                               &                     &                                         &                &                    &                           & \checkmark              & \checkmark                   &                                        & \checkmark                   &                   &                         &                              &                  & Zero and Negative Sequence Limits     &                 &    & \checkmark   &            & Fixed             \\ \hline
\citep{mgrj15}     &                                 &                     &                                         & \checkmark              &                    & Storage                   & \checkmark              &                     & \checkmark                                      & \checkmark                   &                   & \checkmark                       &                              &                  &                                       &                 &    & \checkmark   &            & Scenarios         \\ \hline
\citep{mgrj04}      &                                 &                     &                                         & \checkmark              & \checkmark                  &                           & \checkmark              & \checkmark                   & \checkmark                                      & \checkmark                   & \checkmark                 &                         & \checkmark                            & \checkmark                &                                       & \checkmark               &    &     &            & Analytical        \\ \hline
\citep{mgrj14}     & \checkmark                               &                     &                                         &                & \checkmark                  &                           & \checkmark              &                     &                                        &                     & \checkmark                 &                         &                              &                  &                                       & \checkmark               & \checkmark  & \checkmark   & AMOHSA     & Fixed             \\ \hline
\citep{mgrj49}     &                                 &                     &                                         & \checkmark              &                    & Storage                   & \checkmark              & \checkmark                   & \checkmark                                      & \checkmark                   & \checkmark                 &                         & \checkmark                            &                  &                                       &                 &    &     & SAMCSA      & Scenarios         \\ \hline
\citep{mgrj29}     &                                 &                     &                                         & \checkmark              &                    &                           & \checkmark              &                     & \checkmark                                      &                     & \checkmark                 & \checkmark                       &                              &                  & EV Related Constraints                &                 &    & \checkmark   &            & Fixed             \\ \hline
\citep{mgrj01}      &                                 &                     & \checkmark                                       &                &                    & Load Shedding             &                & \checkmark                   & \checkmark                                      & \checkmark                   &                   &                         &                              &                  &                                       &                 & \checkmark  & \checkmark   &            & Fixed             \\ \hline
\end{tabular}
\end{table}
 \end{landscape}
%Table of Papers
%Explain each factor in a little detail: objectives, constraints, power flow, how decision variables are modelled, methods, additional factors considered
\section{Objectives Used} \label{sec:obj}
Following are the various objectives which the researchers try to optimize:\\
\textbf{Power Losses/Total Energy Losses in a Period}: This is a straightforward objective, and it relates to the economics (less losses would mean less power would have to be produced for same loads being fed) and making system more energy efficient, which in a way also relates to increasing sustainability of the system. It is suggested as one of the proxies for cost of system as$\sum\limits_{j\in \xi_L} I_{j}^*Z_{j}I_{j}$, where the summation is over the set of lines in \citep{mgrj02}. And then this cost is minimized, as cost associated with the power losses. Or in some cases losses over a period are considered, by minimizing $\left(t\sum\limits_{h=1}^H\sum\limits_{b=1}^BR_b\left(I_{bh}\right)^2\right)$, where $h$ runs over various time periods, and $b$ over various lines in the system, like \citep{mgrj52}. In \citep{mgrj51} the losses are estimated in a bit different way: first, a loss coefficient is defined as the ratio between the average and peak power loss, and, multiplying it by peak power loss, we can get an estimate of power losses in a period/day. There are also some seasonal losses which can be considered. However \citep{mgrj04} argues that power losses may not be the best objective because it ignores that the power losses are very much dependent on the voltage profile, which may not be as uniform in islanded microgrid, and that to generate profile with minimum losses (high voltages and low currents in feeders) in a microgrid only using DERs may be quite costly compared to having a little more losses.\\
\textbf{Operating Cost of the microgrid}: This is another straightforward objective, where the operating cost of the microgrid is minimized over the particular period. This operating cost can include various factors. they are $C_{PCC}\left(V_{PCC}I_{PCC}^*\right)$, which corresponds to the cost of the power drawn from the main grid from the point of common coupling(\citep{mgrj07}), $\sum\limits_{i\in\xi_{DG}}C_{DG,i}\left(P_{DG,i}\right)$ or the cost to operate various DERs in the microgrid which are dispatchable (for nondispachable DERs like PV or wind, a fixed cost can be considered (\citep{mgrj49})), $\sum\limits_{j\in\xi_L}\alpha_jI_j$, to account for the maintenance and security costs of the lines (\citep{mgrj02}), cost of switching (fixed per one switching) (\citep{mgrj15}) etc. The costs may also include effects of capital investment on a resource (\citep{mgrj15}). For DGs startup costs can also be included. For all DERs maintenance costs can be included. Also, for DERs burning fuels, there may be associated emission costs in addition to the fuel costs(\citep{mgrj15}). Or there might be additional cost terms according to specific DERs. Customer service interruptions can also be included in operation cost as a lost revenue/customer damage index (\citep{mgrj49}). An objective similar to operating cost is benefit, but it also includes revenue generated by selling electricity to the customers and to the grid, as used in \citep{mgrj15}.\\
\textbf{Vulnerability Index}: In order to make system more stable/reliable, vulnerability index can be minimized. Vulnerability index measures how vulnerable the system is to disturbances. There are two components in vulnerability index. The first one is structural component, which comments about how pivotal a given element is to the system in terms of the structure, so that remote and peripheral components are less vital in this sense than central components. The second factor is operational factor, which speaks about how easily that component can go out of its operating limits, or how close to its operating limits it is working. The vulnerability index is the multiplication of these two terms. Usually the structural term comes from what is known as `betweenness' of that node/line. If the probability of failure of a component (line/bus) is different for every component, then these betweenness indices can be weighted by these probabilities to get a more accurate measure of structural vulnerability index. The operational factor of a node is related to the voltage deviation from its nominal/specified value. The operational factor of a line represents the overload of that line or the flow of the line compared to the allowed flow limit, both in terms of real and reactive power. Sometimes the effect of change of flow in one line on the other lines or effect of change of voltage  of one node on the voltage of other nodes is also considered (\citep{mgrj57}). One more factor which can be considered system wide is the frequency deviation from its nominal value (\citep{mgrj16}).  Vulnerability index of all the elements in the microgrid is added to give the vulnerability index of the microgrid, which is then minimized. In terms of constraints also, limits on vulnerability indices can be put in place. Such a vulnerability index can be particularly valuable since microgrid may be acting in islanded mode and not have a seemingly infinite bus in terms of PCC to balance these deviations out.\\
\textbf{Supplying maximum load/priority loads}: Particularly in islanding, it may turn out that the loads are greater than the available generation, and in this case feeding the maximum amount of loads/feeding the sensitive and priority loads may be important. This can be formulated as $\sum\limits_{k\in \xi_D}w_kx_kP_{L_k}$, where $\xi_D$ is the set of buses where there are shedable loads, $w_k$ are the weights assigned to loads according to the priorities, and $P_{L_k}$ are the load powers (\citep{mgrj03}). Note that this also creates a new set of decision variables, which are associated with loads and are used to shed those loads or keep them connected. This can also be done in presence of faults eg. considering certain DERs and/or lines to be out of service (as done in \citep{mgrj01}).\\
\textbf{Loadability of a system/Voltage Stability Index (VSI)}: Since the microgrid may not be connected to the main grid always, it is important to maintain a stable voltage profile within the microgrid. Hence this objective tries to maximize the ``distance" between the situation of voltage collapse from current operating conditions (\citep{mgrj04}). There are two types of bifurcations, static bifurcation point, at which there's a saddle point in the power flow equations of the microgrid, and `limit induced bifurcation, where the equilibrium can suddenly disappear after a limit has been hit, for example the generation limit of a DER. The difference between maximum and current load factor is minimized in this case. The voltage stability index can also be defined in other ways, for example, based on algebraic modification of power flow equations (\citep{mgrj14}).

\section{Constraints Used}
\textbf{Radiality}: Most of the papers stress that the system has to be radial, but a weakly meshed system is allowed in some papers like \citep{mgrj02}. The reasons behind radiality are that it simplifies protection coordination and operation, is cheaper to build, etc. \citep{mgrj04}. This constraint is converted to be a part of the objective function in \citep{mgrj02}, see in section ~\ref{sec:obj}. Many times the configuration is simply supposed to be from a particular set containing all compatible radial configurations. ($g\in G_k$). Other way is to check radiality for the obtained solutions in the intermediate steps of running the optimization, by a formula like $N_{loops} = N_{lines}-N_{buses}+1$, and reject or modify the ones with a loop. Most papers include radiality as a constraint, but it can be forced to become a part of objective function to be minimized, by using convex Lasso-type regularization term 
%[read about it]
, as $w_{rad}\sum\limits_{j\in\xi_R}||x_j||_2$, where $\xi_R$ is the set of lines which can be closed/opened (which take part in reconfiguration), and the norm of $x_j$ will be positive when that particular line is closed, and zero when that line is open. This works as a penalty factor in the objective function, and the weight corresponding to this function ($w_{rad}$) needs to be chosen such that considering the general values the objective function is taking without this term, the weight factor will be high enough to keep this summation low, which will result in lower lines being closed, and hence a radial or weakly meshed system.\\
\textbf{Power Flow equation}: This constraint comes from the power flow balance at each bus/node, and is a very standard concept. Now, almost all papers either confirm that the power flow is satisfied after computing all the decision variables as a constraint (e.g. \citep{mgrj01}) , or these equations are a part of computing the objective function.\\
\textbf{Thermal or Current Limits of the lines}: This constraint limits the flow through a particular line, and is also fairly standard in power systems. It can be denoted by $|I_j|\le |I_j^{max}|$, where $j$ denotes a particular line. In case of droop controlled DERs, their droop characteristics (since they depend on the local voltage and frequency) need to be included in the power flow equations to be used as a constraint(\citep{mgrj04}) . This limit can also be based on real power flow (\citep{mgrj16}) or apparent power flow (\citep{mgrj29}) through the line in stead of current through the line.\\
\textbf{Probability of feeding all loads being high}: This is a very specific constraint used in \citep{mgrj02}. Here, the net power leaving any node $j$ should be less than load minus generation for the load to be fed successfully, and the constraint denotes the probability of this happening for all the nodes/buses in the system to be high.\\
\textbf{Node Voltage Limits}: This constraint is similar to the line flow limits, and says that the voltage magnitude should be kept between certain values, which can be given as $|V_j^{min}|\le|V_j|\le|V_j^{max}|$.\\
\textbf{Generation Limits of DGs}: There might be a minimum and maximum limit to how much a particular DG can pump in to the system, so it is simply written as $P_G^{min}\le P_G\le P_G^{max}$ (\citep{mgrj03}). Sometimes limits on reactive power are used as well (\citep{mgrj57}). Sometimes generation limits and ramp rates are implemented not explicitly in the optimization problem, but in the corresponding dispatch/OPF problem which is solved along with the reconfiguration problem. Sometimes instead of putting limits on individual $P$ and $Q$, a limit is put on $S$ instead (\citep{mgrj14}). Also, some spinning/operational reserve might be included here depending on the type of DER and the overall microgrid (\citep{mgrj07}).\\
\textbf{Limit on the number of switching instances}: The operation of switching does include an operational cost associated with it, but the reliability and life of switch is also related to or measured in terms of how many switching operations it can undergo before it fails. Now with reconfiguration, such switches are bound to be used much more frequently, and so it makes sense to limit the number of switching operations in a day in order to keep the life of the switches sufficiently long. Also, where the switches are not automated, the operators have to manually go where the switch is and operate it, this can be another reason to minimize the number of switching operations. This can be expressed as $N_{switch}\le N_{max}$, but depending on the problem formulation, there may be a few more steps needed to calculate the number of switching operations $N_{switch}$ (\citep{mgrj49}). It can also be put as an objective to minimize the number of switching operations by minimizing the norm of the difference of decision variable subvector corresponding to switches, as done in \citep{mgrj04}\\
Load Generation Balance: For especially microgrids which are to be operated in islanded mode, this constraint is very important, and it just says that the generation is greater than or equal to the load/usage (\citep{mgrj03}). Sometimes power loss is added to the loads and compared with the generation to impose this constraint (\citep{mgrj07}).\\

\section{Techniques Used} \label{sec:tech}
%\textbf{Linear Programming}: This technique is generally not used in the reconfiguration problem since the problem is mixed integer linear problem. To convert it to linear problem, some special modifications to objective function and constraints are needed. This is a standard optimization technique having multiple methods to solve/implement. This method notably can be seen in \citep{mgrj02}.\\
%Rule based/inhouse [mgrj52, do later]\\
\textbf{Genetic Algorithm (GA)}: This is a population based method, and is inspired by genetic evolution, so it begins with a set of individuals which are random solutions satisfying the constraints. Then in each iteration, the population is ranked according to the fitness value or value of the objective function for each individual, and certain number of individuals with poor fitness are rejected. After that `children' or new individuals are created to create new population, by the processes such as mutation (in which individual decision variables in a parent are randomly changed to create children), crossover (in which decision variables from two parents are taken and mixed to form children) and selection (in which parents which have a very good value of fit are retained). These processes rely on random numbers to work (hence it is a heuristic method). This is a very popular method and many times new methods are compared with GA (e.g. \citep{mgrj57}).\\
\textbf{Particle Swarm Optimization (PSO)}: This is a population based heuristic technique of optimization. It begins with a set of solutions, and then keeps the decision variables changing so as to emulate a flock of birds who wish to find prey. The movement of a particular ``solution" is determined by several factors: the weight of that particular particle, or inertia, the difference between the best value achieved by that particular particle and its current value and the difference between the global best value achieved between all particles and current situation. The weights to these parameters can be tuned. Such iterations go on either till the optimum solution is reached or maximum number of iterations is reached. This is a very popular method, and in fact many times papers compare their method with PSO and GA to assess them (e.g. \citep{mgrj16}).\\
There are more heuristic methods as well, used in different papers. These all have their advantages and disadvantages, and these methods are nicely compared in \citep{mgrrev01}.\\
\textbf{Nondominated Sorting Genetic Algorithm (NSGA 2)}: This algorithm is very popular to be used for multiobjective problems. It works somewhat similar to genetic algorithm, and there are steps such as mutation, crossover and selection, but where it differs is that instead of having a single fitness function and ranking solutions according to it, here, there are multiple fitness functions, and the ranking used is nondominated sorting. In it, instead of sorting according to a particular fitness function value, the solutions are segregated into several nondominated fronts similar to a pareto front, where no solution in each nondominated front is better than the other solutions in that nondominated front in terms of all the fitness functions. By ranking solutions according to their nondominated front, solutions from lower nondominated front are rejected and the genetic algorithm is implemented accordingly. (e.g. \citep{mgrj14})\\ 
%[Don't refer, but http://ieeexplore.ieee.org/abstract/document/996017/]\\
There are other multiobjective heuristic solvers as well (like adaptive multiobjective harmony search algorithm)\citep{mgrj42}, but NSGA2 seems to be very popular. 

\section{Uncertainty Model}
Scenario based constraints are used in \citep{mgrj02}. In this paper, there is a constraint of probability of feeding all the loads being greater than a positive compatible real number. In taking uncertainty into account, the authors of that paper break this continuous constraint into a number of cases or scenarios, selected at random, and assume that if for these cases the constraint is not violated, then it will not be violated.\\
One can also solve the problem for different scenarios (for each scenario it becomes a problem without uncertainty, so is solvable) and then aggregate these solutions to end up with a single solution. The scenario generation usually is based on random number generation, and uses techniques like Monte Carlo \citep{mgrj15} or roulette wheel mechanism\citep{mgrj49}, while the aggregation is based on the expected value we calculate in probabilistic situations. Sometimes generated scenarios are reduced in number by combining similar scenarios to get a smaller number of diverse scenarios so as to reduce the redundant computations (\citep{mgrj49}).\\
Another way uses the forecasts provided to the microgrid and the calculations are performed on these forecasted values (\citep{mgrj07}) . Here, the forecasts can give us some ballpark values, and then these values are assumed to be true. This might work in very low penetration area (where majority of the system is without uncertainty) , but essentially fails to capture the essence of uncertainty. In one scenario, however, this might work: the forecasts can be pretty accurate if very small time scale is considered (a few minutes etc.) (\citep{mgrj57}) , and hence if real time calculations are being performed, this might be better because at the expense of lost accuracy (due to ignoring probabilistic nature of the problem), the calculations needed to be performed are much less.\\
Alternatively, if we take a theoretical approach, we can  analytically use probability in the objective function and/or constraints. In this approach, instead of generating scenarios and then later on combining them in form of expected value, analytically objective function itself is the expected value of that particular parameter, for example, operating cost (\citep{mgrj04}) . This makes the problem significantly complex, since now one has to incorporate and combine the probabilities of loads and various DERs, which can be challenging in itself, and also since the probability density functions of all the loads and DERs may not be available with the required accuracy.
\section{Additional Considerations}
\textbf{Storage}: Adding storage can have impact on the objective, as well as some added constraints corresponding to it. If cost is used as an objective function, then there might be a term corresponding to the operation of storage added to the cost function. A factor of cost to account for the degradation of storage device (particularly batteries) can also be added (as done in \citep{mgrj49}). The constraints may include the energy balance in the storage device ($E_{t+1} = E_t + t*P_t$, where $E$ corresponds to the energy stored in the storage system, and $P$ corresponds to the power exchanged by the storage system with the microgrid, and $t$ is the period of time considered), the minimum and maximum charge limits ($E^{min}\le E_t\le E^{max}$) and maximum charging/discharging rate, limited by the current charge level ($-\eta^{dis}E_t\le P_t\le \eta^{ch}\left(E^{max}-E_t\right)$)  or absolute limits ($-P^{disch,lim}\le P_t\le P^{ch,lim}$)etc. This also adds the decision variables corresponding to the power exchanged with the microgrid in a particular period. Also, when storage devices are considered, usually multiperiod objective functions are used, which include data from multiple periods. In case of multiperiod considerations, additional constraints such as minimum/maximum values of energy stored in the storage systems at the beginning of each day/at specific times of day can also be specified as constraints (e.g. in case of high PV penetration, since there is not much generation in the night, the operator might want to keep the storage levels high at evening, or if an electric vehicle is used as storage device, it may have energy stored requirements for other purposes). For specific storage systems, there may be more constraints (such as electric vehicles \citep{mgrj49},\citep{mgrj29})\\
\textbf{Three Phase Voltage Unbalance}: Since most of the residential loads are single phase, it is important to balance the loads on each phase of a system, and this can be traced in terms of zero and negative sequence components in the voltage. In terms of objective function, the zero and negative component have to be minimized (or we can use zero and negative sequence voltage factors, which are the root mean square of zero/negative voltage component divided by the positive sequence component for each phase). The phase balancing can be achieved by using connection schemes for three phase buses wisely. This adds a couple of decision variables into the problem though. Also, this can add limits to allowed values for zero and negative sequence components, and that will come in the constraints (\citep{mgrj51}) .\\
\textbf{Responsive Loads}: Responsive loads or other forms of demand side management can be added to the optimization problem. This will result in an additional cost term (if the objective function is cost), several additional decision variables, as well as constraints related to the demand side management terms, such as in \citep{mgrj29}.\\
\section{Dealing With Multiobjective Problems}
%combining of objectives
While dealing with multiple objectives, the question of how to combine the objectives arises. There are two possible approaches here. One is to combine all the objectives in a single objective function, say as a weighted addition of all these functions, and this results in a single solution and then most of the techniques used above can be used for the optimization. However, the solution achieved can be very sensitive in this case to the weights given to different objectives, and choosing such weights is often not straightforward. The other way is to use a multiobjective optimization and generate a pareto front as a solution. Pareto front means that the solutions are such that there exists no other solution which has values of all the objective functions better than a pareto solution. In this case instead of getting a single solution, we get a set of solutions, and have to then rank these solutions in order to zero in on one solution (\citep{mgrj52}) . Although while solving we are getting rid of weights to different objective functions, here also while ranking the solutions some kind of weightage must be given to the different objective functions. The techniques used for multi-objective optimization are different from those used for single objective optimization, and a very popular one is NSGA2, mentioned above in techniques section ~\ref{sec:tech}. There are some other techniques as well, such as AMOHSA (\citep{mgrj14}) and can be found in \citep{mgrj42}.
%\section{Other Aspects}
%Real time vs computer simulation [mgrj03]
%Transfer simulation [mgrj23]



%%% Local Variables: 
%%% mode: latex
%%% TeX-master: "../mainrep"
%%% End: 
